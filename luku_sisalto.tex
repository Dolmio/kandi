\section{Johdanto}

Tämä kandidaatintyö käsittelee menetelmiä, joilla silmän fiksaatioita
voidaan tunnistaa katseentunnistuslaitteiden keräämästä datasta. Katseentunnistuksen tarkoituksena on selvittää minne ihminen on katsomassa. On yleisesti hyväksyttyä \citep[s.33]{munn2008}, että ne alueet, joihin ihmisen katse on suuntautunut ovat alueita, jotka ovat jollakin tapaa tärkeitä tai kiinnostavia katsojalleen. 

Analysoitaessa katseentunnistuslaitteden keräämää dataa jaetaan kerätyt datapisteet usein joko fiksaatioiksi tai sakkadeiksi. \citep[s. 71]{salvucci2000} Fiksaatio voidaan määritellä tarkoittamaan aluetta, johon katse pysähtyy \citep[s. 71]{salvucci2000} ja jonka aikana tapahtuu kognitiivista prosessointia, joka mahdollistaa ``näkemisen''  ihmiselle. \citep[s. 881]{Blignaut2009}

Fiksaatoiden tunnistaminen on usein ensimmäinen vaihe katseentunnistuslaitteen keräämän datan analysointia ja sen tarkoituksena on helpottaa korkeamman tason analyysin tekemistä identifioimalla olennaiset osat datajoukosta yhdistämällä datapisteitä fiksaatioiksi ja seulomalla siitä pois sakkadit, eli nopeat siirtymät fiksaatioista toiseen. \citep[s. 18]{mould2012}

Katseentunnistuslaitteiden ja usein niiden mukana tulevien valmiiden katsedatan analysointiohjelmistojen käyttäminen tutkimuksessa on tehnyt helpoksi aliarvioida merkityksen, joka käytetyllä fiksaation tunnistusalgoritmilla on lopulliseen analyysiin datasta. \citep[s. 111]{shic2008}
Tämän työn tavoitteena onkin auttaa katseentunnistuksesta kerättyä dataa hyödyntävää tutkijaa ymmärtämään taustalla käytetyn fiksaatioiden tunnistusalgoritmin merkitystä vastaamalla seuraaviin kysymyksiin:
\begin{itemize}
	\item Miten fiksaatio määritellään?
	\item Millaisia menetelmiä silmän fiksaatioden tunnistamiseen on olemassa?
	\item Millä tavalla näitä menetelmiä voidaan luokitella?
	\item Mitä asioita tulee ottaa huomioon käytettävän menetelmän valinnassa?
	\item Kuinka fiksaatiodataa voidaan yhdistää videokuvaan ihmisen näkökentästä?
	\item Kuinka esitellyt fiksaationtunnistusmenetelmät vertautuvat kaupalliseen toteutukseen?
\end{itemize}

Työ on rajattu niin, että siinä ei tulla käsittelemään erilaisia menetelmiä, joilla katseentunnistuslaite voi kerätä datansa. Työssä ei myöskään tulla selvittämään syvällisesti minkälaisia sovelluksia silmän liikedatan käytölle on olemassa. Työssä tullaan yllä esitettyjen tutkimuskysymysten mukaisesti keskittymään fiksaationtunnistusalgoritmeihin sekä siihen kuinka fiksaatiodataa voidaan yhdistää videokuvaan ihmisen näkökentästä.

Tutkimuskysymysten ratkaisemiseksi tehtiin kirjallisuuskatsaus tutkimukseen, jota havaitsemisen, psykologian ja konenäön aloilla on tutkimuskysymyksiin liittyen tehty. Tämän lisäksi työ sisältää kokeellisen osuuden, jossa verrataan kirjallisuudesta löydettyjen algoritmeja kaupallisen katseentunnistuslaitteen mukana tulleen ohjelmiston tuottamaan fiksaatiodataan. Vertailemalla kirjallisuudesta löytyviä menetelmiä kaupalliseen toteutukseen selvitetään kuinka lähelle kaupallisen toteutuksen tuloksia kirjallisuudesta löytyvillä menetelmillä on mahdollista päästä sekä kuinka haastavaa menetelmiä on toteuttaa. Näin tutkijan on mahdollista evaluoida, missä tapauksissa oman implementaation toteuttaminen olisi järkevää.

Yllä esitetyt tutkimuskysymykset tukevat työn rakennetta niin, että ne tullaan ratkaisemaan siinä järjestyksessä kuin ne on esitetty. Johdannon jälkeisessä luvussa ``Silmän liikeanalyysin tutkimus''  esitellään ja selitetään terminologiaa, jota tutkimuksen aihepiiriin kuuluu sekä esitetään kuinka tietoa tutkimukseen on haettu. VARMAAN JOTAIN MUUTA KANSSA...

Tämän jälkeen kolmannessa luvussa ``Silmän liikeanalyysin algoritmit''  esitellään kirjallisuudesta löytyneitä algoritmeja ja vertaillaan niitä toisiinsa kirjallisuuden perusteella. Tämän lisäksi luvussa esitellään taksonomia, jonka perusteella algoritmeja voi luokitella. Luvun perusteella lukija saa käsityksen siitä millaisia liikeanalyysin algoritmit tyyppillisesti ovat ja mitkä ovat kunkin esitellyn algoritmin vahvuudet ja heikkoudet.

Luvussa neljä ``Fiksaatioiden tunnistaminen liikedatasta''  sovelletaan kirjallisuudesta löydettyjä algoritmeja käytäntöön ja verrataan niiden tuloksia kaupallisen toteutuksen tuottamaan dataan. LISÄÄ TARVITAAN

Lopuksi johtopäätöksissä...
 

---------------------------------------------------------------
\section{Silmän liikeanalyysin tutkimus}

\section{Silmän liikeanalyysin algoritmit}
\subsection{Yksinkertainen nopeusraja-arvoon perustuva tunnistusalgoritmi I-VT}
I-VT fiksaationtunnistusalgoritmi on kaikista kirjallisuudesta esiin tulleista algoritmeista yksinkertaisin ja helpoin ymmärtää. Algoritmin ideana on, että lasketaan jokaisen datapistevälin nopeus ja pidetään pistevälejä joiden nopeus on pienempi kuin asetettu nopeusraja-arvo fiksaatioina ja pistevälejä joiden nopeus on suurempi kuin raja-arvo sakkadeina. Pistevälin nopeus on sama kuin pistevälin pisteiden etäisyys toisistaan, jos oletetaan, että datapisteiden välinen aika on vakio. Kun pistevälien nopeudet on laskettu, yhdistetään algoritmissa peräkkäiset fiksaatiopisteet fiksaatioryhmiksi ja hylätään sakkadipisteet. Lopuksi jokainen fiksaatioryhmä muutetaan muotoon <x,y,d,t>, jossa x:n ja y:n arvot lasketaan ryhmän keskiarvosta kummallekin muuttujalle. T:lle annetaan aika, joka kuului ensimmäiselle fiksaatioryhmään kuuluvalle pisteelle ja d:lle koko fiksaatioryhmän kesto. \citep[s. 73]{salvucci2000}

 Jotta I-VT algoritmia voisi käyttää tarvitsee sille määritellä vain yksi parametri, nopeusraja-arvo. Mikäli silmän etäisyys havainnoitavaan kohteeseen tunnetaan voidaan nopeusraja-arvo johtaa kulmanopeudesta, joka silmällä on suhteessa katsottavaan kohteeseen. Esimerkiksi \citet[s. 1099]{itti2005} on käyttänyt kulmanopeuden raja-arvona 20 astetta / sekunti niin kuin myös \citet[s. 73]{salvucci2000}:n mukaan \citep[s. 103-111]{megaw1984}.
Mikäli etäisyyttä havainnoitavaan kohteeseen ei tunneta niin nopeusraja-arvon määrittämisessä on vain käytettävä eksploratiivista data analyysia aineistoon, jossa otetaan huomioon mm. näytteenottotaajuus.

I-VT on yksinkertainen implementoida kuten liitteestä \textbf{\nameref{sec:IVT-implementation}}  voidaan nähdä. Koska algoritmi käsittelee vain pientä osaa datajoukosta kerralla on se erittäin tehokas ja soveltuu hyvin reaaliaikaiseen datan analysointiin. \citet[s. 76]{salvucci2000} Suurimmat ongelmat algoritmilla esiintyvät silloin kun pistevälien nopeudet ovat lähellä raja-arvoa. Tällöin syntyy herkästi paljon fiksaatioryhmiä, joissa on vain muutamia peräkkäisiä pisteitä, jolloin näyttää helposti siltä, että fiksaatiota on todellista enemmän. Tätä ongelmaa on kierretty esimerkiksi ryhmittämällä tarpeeksi lähellä toisiaan olevat fiksaatioita yhteen katsealueiksi. \citep[s. 329]{just1980} Toinen keino ongelman kiertämiseksi on vaatia sakkadeilta minimikestoa. \citep[s. 103-111]{megaw1984}

\subsection{Dispersioraja-arvoon perustuva tunnistusalgoritmi I-DT}
I-DT tunnistusalgoritmin perustana on havainto siitä, että koska fiksaatiopisteillä on matala nopeus on niillä taipumus kasaantua lähelle toisiaan. Algoritmi



\label{sec:esimluku}

 --------------------------------------------------------------------

\section{Johtopäätökset}

Loppuluku päättää työn. Luvun nimi on tyypillisesti ``yhteenveto'' tai
``johtopäätöksiä''. Valitse se otsikko, joka tuntuu sopivammalta työsi
luonteeseen. Joka tapauksessa loppuluku sisältää niin työn yhteenvedon
kuin johtopäätöksiä työn tulosten perusteella. Pääajatus on antaa
lukijalle selvä kuva siitä, miten johdannossa asetettuihin
tavoitteisiin työssä vastattiin.

Käsittele loppupuvussa seuraavia asioita (jotakuinkin tässä järjestyksessä):
%
\begin{itemize}
  \item Muistutus työn tavoitteista (sidoksisuus johdantoon)
  \item Päätulokset kootaan yhteen, pohditaan niiden merkitystä
  \item Suositukset konkreettisiksi toimenpiteiksi (``Mitä sitten?'' 
Nyt kun käytössä on tämän työn myötä tullut tieto, 
mitä se nyt tarkoittaa tälle asialle/alalle.)
  \item Tulosten soveltuvuus, käyttöön liittyvät rajoitukset
  \item Jatkotutkimustarve 
(``Tulevaisuudessa olisi mielenkiintoista selvittää...'' tms.)
  \item Työn onnistumisen arviointi 
(Huom! Älä arvioi omaa kirjoitusprosessiasi vaan tekemääsi tutkimusta)
\end{itemize}

% --------------------------------------------------------------------

